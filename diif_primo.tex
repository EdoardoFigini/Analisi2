\section{Equazioni differenziali}

\begin{definition}[Equazione differenziale]
    Si definisce equazione differenziale di ordine $n$ un’uguaglianza del tipo
    \begin{equation}
        \boxed{F(x, \, y, \, y', \, y'', \, \dotsc, \, y^{(n)}) =  0} 
    \end{equation}
    dove 

    \begin{itemize}[noitemsep,topsep=0pt]
        \item [$i$)] $y$ dipende da $x$ → $y = y(x)$ è la funzione incognita
        \item [$ii$)]$F$ è una funzione assegnata di $n + 2$ variabili
        \item [$iii$)] $x$ è la variabile \emph{indipendente}
    \end{itemize}
\end{definition}

\begin{definition}[Soluzione dell’equazione differenziale]
    Viene detta soluzione dell’equazione differenziale $F(x, \, y, \, y', \, y'', \, \dotsc, \, y^{(n)}) =  0$ in un certo intervallo $I \subseteq \mathbb{R}$ una funzione $\varPhi : I \to \mathbb{R}$ t.c.
    \begin{equation}
        \boxed{F(x, \, \varPhi(x), \, \varPhi'(x), \, \varPhi''(x), \, \dotsc, \, \varPhi^{(n)}(x) = 0}
    \end{equation}
\end{definition}


\textbf{Verifica:} inserendo $\varPhi$ (e le sue derivate) al posto dell’incognita $y$ (e delle sue derivate), ottengo un’identità ($\forall \; x \in I$ vale)

\begin{definition}[Integrale Generale e Particolare]
    Sia data $F(x, \, y, \, y', \, y'', \, \dotsc, \, y^{(n)}) =  0 \; (E)$, viene detto \emph{INTEGRALE GENERALE} di $(E)$ l’insieme di \emph{tutte le soluzioni} di $(E)$. 

Viene detta \emph{INTEGRALE PARTICOLARE di $(E)$} una \emph{particolare soluzione} di $(E)$.
\end{definition}

\begin{definition}[Problema di Caucy]
    Si dice \emph{PROBLEMA DI CAUCHY} (o “ai valori iniziali”) associata a $F(x, \, y, \, y', \, y'', \, \dotsc, \, y^{(n)}) =  0$ il problema

    \begin{equation}
        \begin{cases}
            F(x, \, y, \, y', \, y'', \, \dotsc, \, y^{(n)}) =  0 \\
            y(x_0) = y_{0,0} \\
            y'(x_0) = y_{0,1} \\
            y''(x_0) = y_{0,2} \\
            \phantom{xx}\vdots \\
            y^{(n-1)}(x_0) = y_{0,n-1} \\
        \end{cases}
    \end{equation}    

    con $y_{0,0}, \; y_{0,1}, \; y_{0,2}, \; \dotsc, y_{0,n-1}\in \mathbb{R}$ fissati.

    L’istante $x_0$ in cui prescrivo i valori di $y, \; y', \; y'', \; \dotsc, y^{(n-1)}$ è sempre lo stesso, \emph{fissato}.

\end{definition}
$n$ condizioni che dicono chi è $y$, insieme a tutte le sue derivate, fino all’ordine $n-1$, in un punto/istante fissato.

\begin{definition}[Forma normale]
    Un’equazione differenziale di ordine $n$ è detta in \emph{FORMA NORMALE} se la derivata di ordine $n$ è esplicitata, ovvero se è della forma
    \begin{equation}
        y^{(n)} = \underbrace{g(x, \,y, \, y', \, y'', \, \dotsc, \, y^{(n-1)})}_{\textnormal{funz. di } n + 1 \textnormal{ variabili}}
    \end{equation}
\end{definition}

\subsection{Equazioni differenziali del I ordine}
Equazioni del tipo
\begin{equation}
    F(x, \, y, \, y') = 0
    \label{primo_ord}
\end{equation}

\begin{proposition}
    L’integrale generale di \textnormal{(\ref{primo_ord})} è dato da una famiglia di funzioni dipendente da un parametro $\overline{c} \in \mathbb{R}$

    \begin{equation}
        \begin{cases}
            F(x, \, y,\, y') = 0\\
            y(x_0) = y_0 \hspace{3em}  \mbox{condizione iniziale}
        \end{cases}
    \end{equation}

    per avere esistenza e unicità della soluzione.
\end{proposition}

Equazioni considerate in \emph{forma normale}, ovvero
\begin{equation}
    y' = f(x, y)
\end{equation}

\subsubsection{Equazioni differenziali a variabili separabili}
Sono nella forma:
\begin{equation}
    y' = a(x)\cdot b(y)
\end{equation}

con $a$, $b$ funzioni \emph{continue}

\underline{\textbf{Risoluzione:}}
\begin{enumerate}
    \item Determinare l’integrale generale dell’equazione $y' = a(x)b(y(x))$
        \begin{enumerate}
            \item   \textbf{Ricerca delle soluzioni costanti:} costanti che annullano $b(y)$ \\
                    sufficiente porre $b(y) \equiv 0$
            \item   \textbf{Tutte le altre soluzioni:}
                    \begin{itemize}
                        \item   Si scrive $y'$ in notazione di Eulero:
                                \begin{equation}
                                    y' = \frac{dy}{dx}
                                \end{equation}
                        \item Si trattano $dy$ e $dx$ come se fossero numeri o funzioni (come se obbedissero alle tipiche regole algebriche) e si portano tutte le $y$ al primo membro e tutte le $x$ al secondo. \[ \frac{dy}{b(y)} = a(x) \cdot dx \]
                        \item   Si integrano entrambi i membri:
                                \begin{equation}
                                    \int \frac{dy}{b(y)} = \int a(x) \, dx 
                                \end{equation}
                    \end{itemize}
                    Si ottiene l'integrale generale dell'equazione integrale (che dipende da una costante $c$) nella forma
                    \[ y(x) = f(x) \]
        \end{enumerate}
    \item   Determinare la soluzione del problema di Cauchy
            \begin{equation}
                \begin{cases}
                    y' = a(x)b(y) \\
                    y(x_0) = y_0
                \end{cases}
            \end{equation}

            Per poter trovare il valore della costante $c$, si pone l'integrale generale in funzione di $x_0$ e lo si pone uguale a $y_0$.
\end{enumerate}

\subsubsection{Equazioni differenziali lineari del primo ordine}
Sono nella forma:
\begin{equation}
    y'(x) + a(x)y(x) =f(x)
    \label{lin}
\end{equation}

dove
\begin{itemize}
    \item $f(x)$ è il termine noto
    \item $a(x)$ è il coefficiente
\end{itemize}

con $a, f$ funzioni continue assegnate
\vspace*{\baselineskip}

Se $f(x) \not\equiv 0$ si chiama “equazione completa” (EC)

Se $f(x) \equiv 0$ si chiama “equazione omogenea associata” (EO)

\begin{theorem}[Formula risolutiva per eq. lineari del primo ordine]
    l’integrale generale di \textnormal{(\ref{lin})} è dato dalla famiglia di funzioni

    \begin{equation}
            y(x) = Ce^{-A(x)} + e^{-A(x)}\cdot B(x)
    \end{equation}

    al variare di $C \in \R$, dove
    
    \begin{equation}
        A(x) = \int a(x) \; dx
    \end{equation}
    \begin{equation}
        B(x) = \int e^{A(x)}f(x) \, dx
    \end{equation}
\end{theorem}

\underline{\textbf{Risoluzione:}}
\begin{enumerate}
    \item Determinare l’integrale generale dell’equazione
    \begin{enumerate}
        \item Si calcola la primitiva $A(x)$
        \item Si applica la formula
        \begin{equation}
            y(x) = \left( \int f(x)e^{-A(x)}\, dx \; + c \right)e^{-A(x)}
        \end{equation}
    \end{enumerate}

    \item Determinare la soluzione del problema di Cauchy 
    \begin{equation}
        \begin{cases}
            y' + a(x)y(x) = f(x) \\
            y(x_0) = y_0
        \end{cases}
    \end{equation}
\end{enumerate}

\subsubsection{Equazioni di Bernoulli}
Sono nella forma:
\begin{equation}
    \left\{
    \begin{array}{l l}
        y' +a(x)y = b(x)y^\alpha, & \alpha \in \R \\
        y(x_0) = y_0, & y_0 > 0 
    \end{array}
    \right.
\end{equation}

\textbf{\underline{Risoluzione: }}
\begin{enumerate}
    \item Si divide per $y^\alpha$ \[ y^{-\alpha}y' + a(x) y^{1-\alpha} = b(x) \]
    \item Si pone $z(x) := y^{1-\alpha}$ 
    \[ \begin{cases}
        z'(x) + (1-\alpha)a(x)z(x) = (1-\alpha)b(x) \\
        x(x_0) = y_0^{1-\alpha}
    \end{cases} \]
\end{enumerate}
