\section{Serie di funzioni}

\begin{definition}
    Sia $I \subseteq \R$ e siano $f_n: I \to R$.\\
    Si definisce 
    \begin{equation}
        \sum_{n=0}^\infty f_n
    \end{equation}
    come la successione delle ridotte $S_N : I \to \R$ con
    \begin{equation}
        S_N(x) = \sum_{n=0}^N f_n
    \end{equation}
\end{definition}

% \subsection{Convergenza puntuale}
\begin{definition}[Convergenza puntuale]
    Sia $I \subseteq \R$ e siano $f_n : I \to \R$.
    Si dice che $\sum_{n=1}^\infty f_n$ converge in un punto $\overline{x} \in \R$ se \[\sum f_n(\overline{x})\] converge.
\end{definition}

\oss{$\sum f_n(\overline{x})$ è una serie numerica.}

% \subsection{Convergenza assoluta}
\begin{definition}[Convergenza assoluta]
    Sia $I \subseteq \R$ e siano $f_n : I \to \R$.
    Si dice che $\sum_{n=1}^\infty f_n$ converge \emph{ASSOLUTAMENTE} su $I$ se, $\forall x \in I$ la serie numerica $\overline{x} \in \R$ se \[\sum |f_n(\overline{x})|\] converge.
\end{definition}

% \subsection{Convergenza totale}
\begin{definition}[Convergenza totale]
    Sia $I \subseteq \R$ e siano $f_n : I \to \R$.
    Si dice che $\sum_{n=1}^\infty f_n$ converge \emph{TOTALMENTE} nell'intervallo $I$ se, $\forall n = 1, 2, \dotsc \exists\, k_n > 0$ t.c.
    \begin{itemize}
        \item [$i$)] \[ |f_n(x)| \leq k_n \mspace{50mu} \forall\, x \in I\]
        \item [$ii$)] \[ \sum_{n=1}^\infty k_n \mspace{50mu} \mbox{converge} \] 
    \end{itemize}
\end{definition}

Per poter verificare la convergenza totale di una serie di funzioni basta verificare che
\[
    \sum \underset{x \in I}{\sup} \mspace{50mu} \mbox{converge}
\]

\subsection{Serie di potenze}
Sono nella forma
\begin{equation}
    \sum_{n=1}^\infty a_n \omega^n
\end{equation}

\begin{definition}[Raggio di convergenza]
    $R \in [0, +\infty]$ t.c. la serie $\sum_{n=1}^\infty a_n \omega^n$ converge per $|\omega| > R$ e diverge per $|\omega| < R$ è detto \emph{RAGGIO DI CONVERGENZA}.
\end{definition}


\subsection{Serie di Fourier}
Sia $f$ una funzione $2 T$-periodica ($f(x) = f(x+2T) \; \forall\, x \in \R, \; T> 0$), e $|f|$ è integrabile. Allora
\begin{equation}
    f \sim S_{[f]}(x) = \frac{a_0}{2} + \sum_{n=1}^\infty (a_n \cos(nx) + b_n \sin(nx))
\end{equation}

\underline{\textbf{Risoluzione:}}
\begin{equation}
    a_0 = \frac{1}{\pi}\int_{-T}^{T} f(x) \,dx
\end{equation}
\begin{equation}
    a_n = \frac{1}{\pi}\int_{-T}^{T} f(x) \cos(nx) \,dx
\end{equation}
\begin{equation}
    b_n = \frac{1}{\pi}\int_{-T}^{T} f(x) \sin(nx) \,dx
\end{equation}

\begin{theorem}[Criterio di Dirichlet]
    Se ${a_n}$ e ${b_n}$ monotone decrescenti, allora $S_{[f]}(x)$ converge puntualmente in tutti i punti di $[0, 2\pi]$, eccetto al più $x=2K\pi$.
\end{theorem}

\begin{definition}[Funzione Regolare a tratti]
    $f:[0, 2\pi] \to \R$ è detta \emph{REGOLARE A TRATTI} se 
    \begin{itemize}
        \item [$i$)] $f$ è limitata su $[0, 2\pi]$.
        \item [$ii$)] si può dividere $[0, 2\pi]$ in un numero \emph{finito} di sottointevalli su cui ciascuno dei quali $f$ è continua e derivabile, inoltre nei limiti di tali intervalli esistono finiti i limiti di $f(x)$ e $f'(x)$.
    \end{itemize} 
\end{definition}

\begin{theorem}[Convergenza puntuale di serie di Fourier]
    Sia $f:[0, 2\pi] \to \R$ regolare a tratti. Allora la sua serie di Fourier converge puntualmente in ogni punto $x_0 \in [0, 2\pi]$ alla \emph{MEDIA DEI LIMITI DESTRO E SINISTRO} di $f$ in $x_0$, cioè $\forall \, x_0 \in [0, 2\pi]$ si ha
    \begin{equation}
        \frac{a_0}{2} + \sum_{n=1}^\infty (a_n \cos(nx_0) + b_n \sin(nx_0)) = \frac{f(x_0^-) + f(x_0^+)}{2}
    \end{equation}
    dove $f(x_0^\pm) = lim_{x \to x_0^\pm} f(x)$.
\end{theorem}