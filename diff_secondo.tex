\subsection{Equazioni differenziali lineari del II ordine}

Sono nella forma:
\begin{equation}
    a(x) y''(x) + b(x) y'(x) + c(x)y(x) = f(x)
\end{equation}

con $a,b,c,f$ funzioni definite e continue in $I$ ed $a \neq 0$.

Le soluzioni di un'equazione differenziale lineare del secondo ordine sono infinite e dipendono da due parametri arbitrari.

\begin{theorem}[Principio di Sovrapposizione]
    Se $y_1$ è soluzione di $ay'' + by' + c y = f_1$ e $y_2$ è soluzione di $ay'' + by' + c y = f_2$, allora 
    \begin{equation}
        y(x) = C_1 y_1 + C_2 + y_2
    \end{equation}
    è soluzione di $ay'' + by' + c y = C_1 f_1 + C_2 f_2$.
\end{theorem}

\subsubsection{Integrale generale delle equazioni lineari del II ordine}

\begin{theorem}[Struttura per le equazioni Omogenee]
    L'integrale generale di $a(x)y'' + b(x)y' + c(x) y = 0$, con $a,b,c$ funzioni definite e continue in $I$ ed $a \neq 0$, è dato da tutte le combinazioni lineari
    \begin{equation}
        y(t) = C_1 y_1(t) + C_2 y_2(t) \mspace{25mu} \forall C_1, C_2 \in \R
    \end{equation}
    dove $y_1$ e $y_2$ sono due soluzioni linearmente indipendenti dell'equazione.
    \label{strut_O}
\end{theorem}

\begin{theorem}[Struttura per le equazioni Complete]
    L'integrale generale di $a(x)y'' + b(x)y' + c(x) y = f(x)$, con $a,b,c,f$ funzioni definite e continue in $I$ ed $a \neq 0$, è dato da tutte e solo le funzioni
    \begin{equation}
        y(t) = C_1 y_1(t) + C_2 y_2(t) + y_P(x) \mspace{25mu} \forall C_1, C_2 \in \R
    \end{equation}
    dove $y_1$ e $y_2$ sono due soluzioni linearmente indipendenti dell'equazione e $y_P$ è una soluzione particolare dell'equazione completa.
\end{theorem}

Per la ricerca dell'integrale generale si considera il \emph{polinomio caratteristico associato}
\begin{equation}
    P(\lambda) = a \lambda^2 + b \lambda + c
\end{equation}
e l'\emph{equazione caratteristica associata}
\begin{equation}
    P(\lambda) = 0
\end{equation}

\underline{\textbf{Risoluzione:}}
\begin{enumerate}
    \item Trovare il polinomio caratteristico
    \item Studiare il segno del discriminante dell'equazione caratteristica
    \item Trovare $\lambda_1$ e $\lambda_2$ e quindi $y_1$ e $y_2$
    \item Tramite il Teorema di Struttura (\ref{strut_O}) trovare l'integrale generale.
\end{enumerate}

Il carattere di $P(\lambda) $ è dato dal segno del discriminante:
\begin{itemize}
    \item Se $\Delta > 0$ \\
    si hanno due soluzioni reali distinte:
    \[  \begin{array}{l l}
        \lambda_1 = \frac{-b - \sqrt{\Delta}}{2a} \\
        \lambda_1 = \frac{-b + \sqrt{\Delta}}{2a} 
    \end{array} \]
    da cui 
    \begin{equation}
        \begin{array}{l l}
            y_1(x) = e^{\lambda_1 t} \\
            y_1(x) = e^{\lambda_2 t} 
        \end{array} 
    \end{equation}  
    La soluzione sarà
    \begin{equation}
        \boxed{y(t) = C_1 e^{\lambda_1 t} + C_2 e^{\lambda_2 t} \mspace{25mu} \forall C_1, C_2 \in \R}
    \end{equation}

    \item Se $\Delta < 0$ \\
    si hanno due soluzioni complesse coniugate
    \[  \begin{array}{l l}
        \lambda_1 = \alpha + i \beta \\
        \lambda_1 = \alpha - i \beta 
    \end{array} \]
    dove \[ \begin{array}{l l} \alpha=-\frac{b}{2a} & \beta=\frac{\sqrt{4ac - b^2}}{2a} \end{array} \]
    Per avere soluzioni reali, si sfrutta la linearità, ottenendo
    \begin{equation}
        \begin{array}{l}
            u_1(t) = e^{\alpha t}\cos (\beta t) \\
            u_1(t) = e^{\alpha t}\sin (\beta t) \\
        \end{array}
    \end{equation}
    La soluzione sarà
    \begin{equation}
        \boxed{y(x) = e^{\alpha t}(C_1 \cos(\beta t) + C_2 \sin(\beta t) ) \mspace{25mu} \forall C_1, C_2 \in \R}
    \end{equation}
    \item Se $\Delta = 0$ \\
    si hanno due soluzioni reali coincidenti
    \[ \lambda_1 = \lambda_2 = -\frac{b}{2a}\]
    Per avere due soluzioni linearmente indipendenti si pone
    \begin{equation}
        \begin{array}{l l}
            y_1(x) = e^{\lambda_1 t} \\
            y_1(x) = t e^{\lambda_2 t} 
        \end{array} 
    \end{equation}
    La soluzione sarà
    \begin{equation}
        \boxed{y(t) = C_1 e^{\lambda_1 t} + C_2 t e^{\lambda_2 t} \mspace{25mu} \forall C_1, C_2 \in \R}
    \end{equation}
\end{itemize}