\subsection{Sistemi di equazioni differenziali lineari}
Un'equazione differenziale di ordine $n$ t.c. 
\[
    y^{(n)} = f(t, y', y'', \dotsc, y^{(n-1)})
\]
ed il problema di Cauchy corrispondente, può essere riscritta nella forma di sistema di equazioni differenziali: 
\begin{equation}
    \left\{
    \begin{array}{l}
        y_1' (t) = y_2(t) \\
        y_2' (t) = y_3(t) \\
        y_3' (t) = y_4(t) \\
        \phantom{y} \vdots \\
        y_n'(t) = f(t, y_1, y_2, \dotsc, y_{(n-1)})
    \end{array} \right.
\end{equation}

\oss{Mentre ogni eq. può essere convertita, non tutti i sistemi possono essere scritti in forma di equazione.}

\es\\
Posta
\[ y'' + ay' + y = 0 \]
si pone $z = y'$:
\begin{equation}        
    \begin{cases}
        y'(t) = z(t)\\
        z'(t) = -az(t) - cy
    \end{cases}
\end{equation}

\[
    \left[  
        \begin{matrix}
            y'(t) \\ z'(t)
        \end{matrix} 
    \right] = \underbrace{\left[
        \begin{matrix}
            0 & 1 \\ -c & -a
        \end{matrix}
    \right]}_{A} \left[
        \begin{matrix}
            y(t) \\ z(t)
        \end{matrix}
    \right]
\]

\underline{\textbf{Risoluzione:}}\\
Dato un sistema:
\[
    \begin{cases}
        y' = ay + bz \\
        z' = cy + dz
    \end{cases}
\]
\[
    \left[  
        \begin{matrix}
            y'(t) \\ z'(t)
        \end{matrix} 
    \right] = \underbrace{\left[
        \begin{matrix}
            a & c \\ b & d
        \end{matrix}
    \right]}_{A} \left[
        \begin{matrix}
            y(t) \\ z(t)
        \end{matrix}
    \right]
\]
\begin{enumerate}
    \item Ricerca degli autovalori $\lambda$ di $A$:
    \begin{equation}
        |A -\lambda I|=0
    \end{equation}
    \[
        \left| \begin{matrix}
            a-\lambda & b \\
            c & d-\lambda 
        \end{matrix} \right| = 0
    \]

    Si ottiene $P(\lambda)$

    \item Per ogni autovalore $\lambda$ si cerca l'autovettore $\vec{\omega}$ corrispondente:
    \begin{equation}
        [A - \lambda I] \cdot \vec{\omega} = \underline{0}
    \end{equation}
    \[
        \left[ \begin{matrix}
            -\lambda & 1 \\
            -c & -a-\lambda 
        \end{matrix} \right] \left[ \begin{matrix}
            \omega_1 \\ \omega_2 
        \end{matrix} \right] = \left[ \begin{matrix}
            0 \\ 0
        \end{matrix}\right]
    \]
    \item Le soluzioni esponenziali di $(EO)$ sono del tipo:
    \begin{equation}
        \vec{u} = \vec{\omega}  e^{\lambda t}
    \end{equation}
    \item L'integrale generale è dato da
    \begin{equation}
        \left[
            \begin{matrix}
                y \\ z
            \end{matrix}
        \right] = C_1 \vec{u_1} + C_2 \vec{u_2}
    \end{equation}
\end{enumerate}

% WRONSKIANA
\subsubsection{Determinante Wronskiano}


% ESPONENZIALE DI MATRICE
\subsubsection{Esponenziale di matrice}


% SIST NON OMOGENEI
\subsubsection{Sistemi non omogenei}