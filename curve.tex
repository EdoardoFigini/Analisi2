\section{Curve Parametriche}

\begin{definition}[Curva parametrica]
    Si chiama \emph{CURVA PARAMETRICA} in $\R^n$ una funzione continua 
    \begin{equation}
        \varphi : I \subseteq \R \to \R^n
    \end{equation}
    dove $\varphi$ è data da $\varphi(t) = (\varphi_1(t),\, \varphi_2(t),\, \dotsc,\, \varphi_n(t))$, che è equivalente a 
    \[
        \begin{cases}
            x_1 = \varphi_1(t) \\
            x_2 = \varphi_2(t) \\
            \vdots \\
            x_n = \varphi_n(t)
        \end{cases}
    \]
\end{definition}


$\varphi(I)$ (l'immagine) è detta \emph{SOSTEGNO DELLA CURVA} $\varphi$.

\begin{definition}[Curve semplici]
    Una curva parametrica $\varphi:[a,b] \to \R^n$ è detta \emph{SEMPLICE} se $\varphi:(a,b) \to \R^n$ è iniettiva.
\end{definition}

\begin{definition}[Curve regolari]
    Una curva $\varphi:[a,b] \to \R^n$ è detta \emph{REGOLARE} se $\varphi \in C^1((a,b))$ e $\varphi'(x) \neq \underline{0}$ $\forall\, t \in (a, b)$.
\end{definition}

\begin{definition}[Curve regolari a tratti]
    Una curva $\varphi:[a,b] \to \R^n$ è detta \emph{REGOLARE A TRATTI} esiste una suddivisione in un numero finitop di sottointervalli $[t_k, t_{k+1}]$ di $[a,b]$ in modo che$\left.\phantom{\frac{1}{1}}\varphi\right|_{[t_k, t_{k+1}]}$ sia regolare $\forall \, k$.
\end{definition}

\subsection{Lunghezza di una curva}
la lunghezza di una curva si calcola con
\begin{equation}
    L(\varphi) = \int_a^b || \varphi' (t) || \, dt
\end{equation}

Dove $\varphi' (t)$ è detto \emph{VETTORE TANGENTE} alla curva $\varphi$ nell'istante $t$ e la sua norma $||\varphi' (t)||$ è detta \emph{VELOCITA' SCALARE} di $\varphi$.

\subsection{Curve equivalenti (Riparametrizzazioni)}
\begin{definition}[Curve equivalenti]
    Due curve $\varphi:[a,b] \to \R^n$ e $\psi :[c,d] \to \R^n$ sono dette equivalenti ($\varphi \sim \psi$) se esiste una funzione $\eta : [a,b] \to [c,d]$, di classe $C^1$ e invertibile (t.c. $\eta'(t) \neq 0 \, \forall t \in [a,b]$), detta "\emph{CAMBIO DI PARAMETRO}" t.c.
    \begin{equation}
        \varphi(t) = \psi (\eta(t)) \mspace{25mu} \forall \, t \in [a,b]
    \end{equation}
\end{definition}

\begin{definition}
    $\varphi$ è detta \emph{RIPARAMETRIZZAZIONE} di $\psi$.
\end{definition}

\subsection{Integrali curvilinei}
\begin{definition}[Integrale curvilineo]
    Sia $\varphi:[a,b] \to \R^n$ parametrizzazione di una curva regolare $\gamma$ con sostegno $\Gamma$.
    
    Sia data $f:A \subseteq \R^n \to \R$, $\Gamma \subseteq A$.

    Si dice \emph{INTEGRALE CURVILINEO} di prima specie di $f$ lungo $\Gamma$ l'espressione
    \begin{equation}
        \int_\gamma f \, dl = \int_a^b f(\varphi(t)) \, ||\varphi'(t)|| \, dt
    \end{equation}
\end{definition}

\subsubsection{Baricentro}
\[
    G = (x_1, \, x_2, \, \dotsc, \, x_n)
\]
Si calcola con
\begin{equation}
    \left\{
    \begin{array}{l}
        x_1 = \frac{1}{\int_\gamma f \, dl} \int_\gamma x_1 \, f\, dl \\
        x_2 = \frac{1}{\int_\gamma f \, dl} \int_\gamma x_2 \, f\, dl \\
        \vdots \\
        x_n = \frac{1}{\int_\gamma f \, dl} \int_\gamma x_n \, f\, dl \\
    \end{array} \right.
\end{equation}
che è la media pesata.

In $\R^2$ è $G = (x_G,\, y_G)$.
\[
    \left\{
    \begin{array}{l}
        x_G = \frac{1}{\int_\gamma f \, dl} \int_\gamma x \, f\, dl \\
        y_G = \frac{1}{\int_\gamma f \, dl} \int_\gamma y \, f\, dl \\
    \end{array} \right.
\]